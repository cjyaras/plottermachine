\documentclass[12pt]{article}

\usepackage[letterpaper, margin=1.0in]{geometry}
\usepackage{appendix}
\usepackage[utf8]{inputenc}
\usepackage[T1]{fontenc}
\usepackage{siunitx}
\usepackage{caption}
\usepackage{moreverb}
\usepackage{parskip}
\usepackage{multirow}
\usepackage{adjustbox}
\usepackage{enumerate}
\usepackage{listings}
\usepackage{graphicx}
\usepackage{pdfpages}
\usepackage[american,straightlabels]{circuitikz}
\usepackage{fancyhdr}
\usepackage[explicit]{titlesec}
\usepackage{xcolor}
\usepackage[many]{tcolorbox}
\usepackage{eqparbox}
\usepackage{hyperref}

\hypersetup{
    colorlinks=true,
    linkcolor=black,
    filecolor=magenta,      
    urlcolor=blue,
}

\pagestyle{fancy}

\renewcommand{\headrulewidth}{0.2mm}
\renewcommand{\footrulewidth}{0.2mm}

\fancyhead[L]{ECE 350}
\fancyhead[R]{Page \thepage}
\fancyfoot[L]{Final Project}
\fancyfoot[C]{Carreiro, Yaras}
\fancyfoot[R]{Page \thepage}

\def\arraystretch{1.5}

\begin{document}
\title{ECE 350 Final Project}
\author{Januario Carreiro and John Yaras}
\date{\today}

\maketitle

\thispagestyle{fancy}

\newpage

%A PDF on Gradescope
%Kind of like a lab report but longer
%Expecting about 8 to 10 pages
%Describe the overall project design and specifications
%Describe input and output
%Changes you had to make to your processor
%Challenges you faced and how you overcame them
%Any circuit diagrams, how you constructed them, and the rationale behind them
%Describe how you tested your project
%Describe your assembly programs or code
%What improvements you would make / new features you would add if you had more time
%Pictures of your project

\begin{figure}[ht!]
\begin{center}
\begin{circuitikz}
\ctikzset{multipoles/thickness=3}
\ctikzset{multipoles/dipchip/width=2}
\draw (0,0) node[dipchip, num pins=16, hide numbers, no topmark, external pins width=0](C){SMD};
\node [right, font=\tiny] at (C.bpin 1) {ENABLE};
\node [right, font=\tiny] at (C.bpin 2) {M0};
\node [right, font=\tiny] at (C.bpin 3) {M1};
\node [right, font=\tiny] at (C.bpin 4) {M2};
\node [right, font=\tiny] at (C.bpin 5) {RESET};
\node [right, font=\tiny] at (C.bpin 6) {SLEEP};
\node [right, font=\tiny] at (C.bpin 7) {STEP};
\node [right, font=\tiny] at (C.bpin 8) {DIR};
\node [left, font=\tiny] at (C.bpin 16) {VMOT};
\node [left, font=\tiny] at (C.bpin 15) {GND};
\node [left, font=\tiny] at (C.bpin 14) {B2};
\node [left, font=\tiny] at (C.bpin 13) {B1};
\node [left, font=\tiny] at (C.bpin 12) {A1};
\node [left, font=\tiny] at (C.bpin 11) {A2};
\node [left, font=\tiny] at (C.bpin 10) {VDD};
\node [left, font=\tiny] at (C.bpin 9) {GND};
\draw (C.bpin 5) 	to[short] ++ (-1,0) node(n1) {};
\draw (C.bpin 6) 	to[short] ++ (-1,0) 
					to[short] (n1);
\draw (C.bpin 7) 	to[short, -o] ++ (-2,0) node[anchor=east] {DIG 7}; % UPDATE
\draw (C.bpin 8) 	to[short, -o] ++ (-2,0) node[anchor=east] {DIG 8}; % UPDATE
\draw (C.bpin 9) 	to[short] ++ (1,0) 
					to[short] ++ (0,-1) node(n2) {}
					to[short] (n2 -| n1) 
					to[short, -o] ++(-1,0) node[anchor=east] {5V}; % UPDATE
\draw (C.bpin 10) 	to[short] ++ (1.6,0) node(n3) {}
					to[short] (n3 |- n2)
					to[short] ++(0,-.6) node(n4) {}
					to[short] (n4 -| n1) 
					to[short, -o] ++(-1,0) node[anchor=east] {GND}; % UPDATE
\draw (C.bpin 11) 	to[short, -o] ++ (2,0) node[anchor=west] {PHASE /A};
\draw (C.bpin 12) 	to[short, -o] ++ (2,0) node[anchor=west] {PHASE A};
\draw (C.bpin 13) 	to[short, -o] ++ (2,0) node[anchor=west] {PHASE /B};
\draw (C.bpin 14) 	to[short, -o] ++ (2,0) node[anchor=west] {PHASE B};
\draw (C.bpin 15) 	to[short] ++ (4,0) 
					to[short] ++ (0,2) node(n5) {}
					to[short, -o] ++ (0,1) node[anchor=south] {GND};
\draw (C.bpin 16) 	to[short] ++ (2,0) node(n6) {}
					to[short] (n6 |- n5) node(n7) {}
					to[C, l_=\SI{47}{\nano\farad}] (n5);
\draw (n7)			to[short, -o] ++(0,1) node[anchor=south] {VDD};
\end{circuitikz}
\end{center}
\caption{Circuit Diagram for Stepper Motor Drivers}
\end{figure}


\end{document}
